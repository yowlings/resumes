% !TEX TS-program = xelatex
% !TEX encoding = UTF-8 Unicode
% !Mode:: "TeX:UTF-8"

\documentclass{resume}
\usepackage{zh_CN-Adobefonts_external} % Simplified Chinese Support using external fonts (./fonts/zh_CN-Adobe/)
%\usepackage{zh_CN-Adobefonts_internal} % Simplified Chinese Support using system fonts
\usepackage{linespacing_fix} % disable extra space before next section
\usepackage{cite}

\begin{document}
\pagenumbering{gobble} % suppress displaying page number

\name{王晓妍}

\basicInfo{
  \email{wangxiaoyan185@126.com} \textperiodcentered\
  \phone{18001366397} \textperiodcentered\
  }


\section{\faGraduationCap\  教育背景}
\datedsubsection{\textbf{北京航空航天大学}}{2012.9 -- 2015.4}
\textit{硕士研究生}\     软件学院, 互联网营销与管理专业, 新媒体运营,互联网营销方向
\datedsubsection{\textbf{天津师范大学}}{2006.9 -- 2010.6}
\textit{学士}\   汉语言文学专业,广告新闻学方向

\section{\faBriefcase\  工作经历}
\datedsubsection{\textbf{梦想直播}}{2016.9 至今}
\textit{市场策划}\ 将梦想直播推广树立成业内黑马形象。主导并负责梦想直播所有对外公关事务及品牌传播、以及危机公关。

\textit{主要营销事件:}
\begin{itemize}[parsep=0.5ex]
  \item{戛纳国旗装事件;}
  \item{融资事件朋友圈传播(梦想首次成功亮相);}
  \item{女主播三里屯醉酒事件;}
\end{itemize}
\datedsubsection{\textbf{上海乐融金融}}{2015.8 -- 2016.9}
\textit{市场推广经理}\ 负责所有市场推广项目、产品立意及设计、金融自媒体平台运营
参与了乐融所有金融产品的立意及设计,包括乐融网站、手机APP、好老板、易车贷、嘀嘀贷等。全程参与与通联支付合作的好老板APP的设计、测试及解决方案。
市场上所有的关于乐融品牌和产品的展示,包括近50篇媒体宣传报道、近30篇微信原创文案、产品宣传页设计及所有与产品后期解决方案。
\datedsubsection{\textbf{中国出版集团}}{2013.1 -- 2015.8}
\textit{记者、代理主编}

 1、负责旗下《中国出版传媒商报》互联网版块。
 
 自兼职到实习到正式,两年来上百篇深度报道、30余万字的文稿撰写之余,还为报社搭线建立与多家企业的合作关系:
如:对优酷土豆古永锵、新浪微博刘新征、爱奇艺龚宇等互联网界领军人物进行的人物专访,被置于网站首页品牌栏目,均为网站带来上百万的流量,同时,在采访之后,顺利与三家企业建立稳定的新闻合作关系、资源互换关系;

对《时尚》集团总裁刘江、《城市画报》主编刘琼雄、安徽出版集团董事长总裁王亚飞等传媒业人物的采访稿件,得到各大媒体的广泛转载,尤其2013年实习期对《时尚》集团刘江的专访、以及参与制作的《时尚》专刊,一直被当做业内范本.

对北大纵横曲严明、新生代市场监测机构肖明超、人民大学新闻传播学院院长喻国明采访后,将各位企业导师、学校导师带入到报社的线下活动中授课,广受好评。

2、微博、微信运营与推广。

半年的努力下,微信粉丝上升近万人。同时,拉来了拍拍文娱对接微店业务的合作。
\datedsubsection{\textbf{天津日报集团}}{2010.6 -- 2012.9}
\textit{都市报记者}\ 曾对导演贾樟柯、音乐人吴立群、作家苏芩、蒋方舟、电商专家黄相如等各行各业知名人士进行采访;稿件及文案撰写;社会名人联络工作
\section{\faCogs\  特色技能}
% increase linespacing [parsep=0.5ex]
\begin{itemize}[parsep=0.5ex]
  \item \textbf{文案撰写}: 天津都市频道剧《十万元之谜》, 剧本天津都市频道剧《项链》剧本
  \item \textbf{互联网营销}: 研究生期间多次参与互联网整合营销实战及企业级互联网营销全案诊断的KPI项目,为驴妈妈旅游网,携程旅游网在内的多家企业撰写完善的互联网营销全案诊断,其中包括:SEO,SEM,UEO,SMO在内的多项技术内容。
  \item \textbf{新媒体运营}: 2014年,为“未名天日语培训学校”撰写新媒体营销方案,为其拍摄、并参与演出日文版《同桌的你》MV,在腾讯视频两周内突破1.2万次播放量。
  \item \textbf{记者出身}:对优酷土豆古永锵、新浪微博刘新征、爱奇艺龚宇等互联网界领军人物进行的人物专访
\end{itemize}

\section{\faUsers\ 项目经验}
\datedsubsection{\textbf{中国科学院软件研究所xlab机器人组组长}}{2016年4月 至今}
\role{xbot服务机器人项目研发}

\begin{itemize}
  \item xbot服务机器人研发的整体规划和任务分配;
  \item xbot移动机器人底盘的运动控制驱动程序开发,ROS程序开发支持;
  \item 完成了ROS环境下集xbot\_driver,xbot\_bringup,xbot\_slam,xbot\_navigation,xbot\_navigoals等多个功能模块于一体的服务机器人定位与导航;
  \item 实现了机器人SLAM与导航在软件博物馆的应用落地,xbot机器人带领参观者参观并同步讲解博物馆内容;
\end{itemize}

\datedsubsection{\textbf{中国科学院软件研究所-腾讯优图合作项目主要负责人}}{2016年4月 至今}
\role{具备主动人脸识别的迎宾机器人}

\begin{itemize}
  \item 实现了迎宾机器人的SLAM,运动规划与导航功能;
  \item 完成迎宾服务机器人在酒店,咖啡厅等场景的应用测试;
  \item 集成了腾讯优图人脸识别应用,实现安卓端的识别与ROS的通信协作;
\end{itemize}

\datedsubsection{\textbf{北京航空航天大学硕士论文研究课题}}{2013年6月 -- 2016年4月}
\role{未知环境中异构多机器人协作探测}

\begin{onehalfspacing}
\begin{itemize}
  \item NAO人形双足机器人运动控制;
  \item NAO与两个双轮差分式机器人协作探测;
  \item ROS环境中仿真实现整体环境探测率98\%以上;
  \item 多机实验中完成机器人协作策略算法和实际运动合作规划;
\end{itemize}
\end{onehalfspacing}

% Reference Test
%\datedsubsection{\textbf{Paper Title\cite{zaharia2012resilient}}}{May. 2015}
%An xxx optimized for xxx\cite{verma2015large}
%\begin{itemize}
%  \item main contribution
%\end{itemize}



\section{\faGratipay\ 获奖情况}
\datedline{\textit{全国大学生数学竞赛一等奖}}{2011年}
\datedline{\textit{优秀学生干部}}{2014年}
\datedline{\textit{优秀研究生}}{2015年}
\datedline{\textit{北京市优秀毕业生}}{2016年}

\section{\faBook\ 发表论文}
EI检索:Peng Wang, Shiyin Qin:
\textit{Heterogeneous multi-robot behavior evolution towards target searching in unknown environment based on reinforcement learning}
2015 IEEE International Conference on Modeling, Algorithm and Artificial Intelligence (MAAI 2015)

\section{\faCommenting\ 个人特点}
\begin{enumerate}
  \item 注重实践.一切算法效果以机器人的实际实验结果为准,在实践中迭代优化;
  \item 注重团队合作与沟通;
  \item 写代码必须理论依据清晰,必要时需要从论文开始研究;
  \item 深知软件框架在持续开发中的重要性;
  \item 以实现具备超人工智能的机器人为个人志向.
\end{enumerate}
\section{\faInfoCircle\ 参考材料}
% increase linespacing [parsep=0.5ex]
\begin{itemize}[parsep=0.5ex]
  \item \textbf{个人GitHub}: https://github.com/yowlings
  \item \textbf{xbot机器人组Github}: https://github.com/XbotGroup
  \item \textbf{技术博客}: https://yowlings.github.io/
  \item \textbf{xbot机器人ROS页面}: http://robots.ros.org/xbot/
  \item \textbf{xbot ROS wiki开源教程}: http://wiki.ros.org/Robots/Xbot?distro=indigo
\end{itemize}

%% Reference
%\newpage
% \bibliographystyle{IEEETran}
\end{document}
